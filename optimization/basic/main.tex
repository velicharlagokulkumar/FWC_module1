\def\mytitle{OPTIMIZATION USING PYTHON}
\def\myauthor{V.GOKULKUMAR}
\def\contact{velicharlagokulkumar@gmail.com}
\def\mymodule{Future Wireless Communication (FWC)}
\documentclass[10pt, a4paper]{article}
\usepackage[a4paper,outer=1.5cm,inner=1.5cm,top=1.75cm,bottom=1.5cm]{geometry}
\twocolumn
\usepackage{graphicx}
\graphicspath{{./images/}}
\usepackage[colorlinks,linkcolor={black},citecolor={blue!80!black},urlcolor={blue!80!black}]{hyperref}
\usepackage[parfill]{parskip}
\usepackage{lmodern}
\usepackage{tikz}
	\usepackage{physics}
%\documentclass[tikz, border=2mm]{standalone}
%\usepackage{karnaugh-map}
%\documentclass{article}
\usepackage{tabularx}
%\usepackage{circuitikz}
\usepackage{enumitem}
\usetikzlibrary{calc}
\usepackage{amsmath}
\usepackage{amssymb}
\renewcommand*\familydefault{\sfdefault}
\usepackage{watermark}
\usepackage{lipsum}
\usepackage{xcolor}
\usepackage{listings}
\usepackage{float}
\usepackage{titlesec}
\providecommand{\mtx}[1]{\mathbf{#1}}
\titlespacing{\subsection}{1pt}{\parskip}{3pt}
\titlespacing{\subsubsection}{0pt}{\parskip}{-\parskip}
\titlespacing{\paragraph}{0pt}{\parskip}{\parskip}
\newcommand{\figuremacro}[5]{
    \begin{figure}[#1]
        \centering
        \includegraphics[width=#5\columnwidth]{#2}
        \caption[#3]{\textbf{#3}#4}
        \label{fig:#2}
    \end{figure}
}

\newcommand{\myvec}[1]{\ensuremath{\begin{pmatrix}#1\end{pmatrix}}}
\let\vec\mathbf
\lstset{
frame=single, 
breaklines=true,
columns=fullflexible
}
\thiswatermark{\centering \put(181,-119.0){\includegraphics[scale=0.13]{iith_logo3}} }
\title{\mytitle}
\author{\myauthor\hspace{1em}\\\contact\\FWC22034\hspace{6.5em}IITH\hspace{0.5em}\mymodule\hspace{6em}Assignment}
\begin{document}
	\maketitle
	\tableofcontents
   \section{Problem}
 A toy company manufactures two types of dolls, A and B. Market tests and available resources have indicated that the combined production level should not exceed 1200 dolls per week and the demand for dolls of type B is at most half of that for dolls of type A. Further, the production level of dolls of type A can exceed three times the production of dolls of other types by at most 600 units. If the company makes a profit of Rs.12 and Rs.16 per doll respectively on dolls A and B, how many of each should be produced weekly in order to maximize the profit?
\section{Solution}
Consider,\\
\begin{table}[H]
 \centering
 \resizebox{\columnwidth}{!}{
 \begin{tabular}{ |c|c|c| } 
 \hline
 \textbf{Description} & \textbf{Parameter} & \textbf{Value}  \\
 \hline
Profit of each type & $\vec{c}$ & $\myvec{12\\16}$ \\
 \hline
 No.of dolls of each type &$\vec{x}$ &$ \myvec{x\\y}$\\
 \hline
\end{tabular}}
 \caption{}
 \end{table}
Objective function 
\begin{align}
z = \max_\vec{x} \vec{c}^T\vec{x}
\end{align}
Constraints
\begin{align}
x+y\leq 1200\\
2y-x \leq 0\\
x-3y \leq 600
\end{align}
Writing all the constraints in the matrix form
\begin{align}
&\vec{p}\vec{x} = \vec{q}\\
&\myvec{1&1\\-1&2\\1&-3}\vec{x}=\myvec{1200\\ 0\\ 600}
\end{align}
By providing the objective function and constraints to cvxpy, we get the the number of dolls of each type should be produced weekly in order to maximize the profit (z) and No.of dolls of each type $(\vec{x})$.\\
From cvxpy, we get
\begin{align}
& \vec{x} = \myvec{800\\400}
\end{align}
\textbf{termux commands :}
\begin{lstlisting}
bash op.sh............using shell command
\end{lstlisting}

\textbf{cvxpy code}:
\begin{center}
\fbox{\parbox{8.5cm}{\url{https://github.com/velicharlagokulkumar/FWC_module1/blob/main/optimization/basic/codes/optimize.py}}}
\end{center}
\end{document}
ent}