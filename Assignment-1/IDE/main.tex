\def\mytitle{IMPLEMENTATION OF BOOLEAN LOGIC}
\def\myauthor{V.GOKULKUMAR}
\def\contact{velicharlagokulkumar@gmail.com}
\def\mymodule{Future Wireless Communication (FWC)}
\documentclass[10pt, a4paper]{article}
\usepackage[a4paper,outer=1.5cm,inner=1.5cm,top=1.75cm,bottom=1.5cm]{geometry}
\twocolumn
\usepackage{graphicx}
\graphicspath{{./images/}}
\usepackage[colorlinks,linkcolor={black},citecolor={blue!80!black},urlcolor={blue!80!black}]{hyperref}
\usepackage[parfill]{parskip}
\usepackage{lmodern}
\usepackage{tikz}
%\documentclass[tikz, border=2mm]{standalone}
\usepackage{karnaugh-map}
%\documentclass{article}
\usepackage{tabularx}

\usepackage{circuitikz}
\usetikzlibrary{calc}


\renewcommand*\familydefault{\sfdefault}
\usepackage{watermark}
\usepackage{lipsum}
\usepackage{xcolor}
\usepackage{listings}
\usepackage{float}
\usepackage{titlesec}

\titlespacing{\subsection}{1pt}{\parskip}{3pt}
\titlespacing{\subsubsection}{0pt}{\parskip}{-\parskip}
\titlespacing{\paragraph}{0pt}{\parskip}{\parskip}
\newcommand{\figuremacro}[5]{
    \begin{figure}[#1]
        \centering
        \includegraphics[width=#5\columnwidth]{#2}
        \caption[#3]{\textbf{#3}#4}
        \label{fig:#2}
    \end{figure}
}

\lstset{
frame=single, 
breaklines=true,
columns=fullflexible
}

\thiswatermark{\centering \put(410,-128.0){\includegraphics[scale=0.09]{iith_logo}} }
\title{\mytitle}
\author{\myauthor\hspace{1em}\\\contact\\IITH\hspace{0.5em}-\hspace{0.5em}\mymodule}
\date{}
\begin{document}
	\maketitle
	\tableofcontents
	\begin{abstract}
	    To Obtain the Boolean Expression for the Logic circuit shown below
	  	\end{abstract}
	  	
	   \begin{circuitikz} \draw
(0,2) node[or port]  (myor1) {}
(0,0) node[and port] (myand) {}
(2,1) node[or port] (myor2) {}
(myor1.out) -- (myor2.in 1)
(myand.out) -- (myor2.in 2);

\node[left] at (myor1.in 1) {\(X\)};
\node[left] at (myor1.in 2) {\(Y\)};
\node[left] at (myor1.in 1)[ocirc] {};
\node[left] at (myand.in 2) [ocirc] {};
\node[left] at (myand.in 1) {\(Y\)};
\node[left] at (myand.in 2) {\(Z\)};
\node[right] at (myor1.out) {\((X'+Y)\)};
\node[right] at (myand.out) {\((YZ')\)};

\node[right] at (myor2.out) {\(F=(X'+Y)+(YZ)\)};
\end{circuitikz}
\begin{center}
Fig. 1
\end{center}

	\section{Components}
  \begin{tabularx}{0.4\textwidth} { 
  | >{\centering\arraybackslash}X 
  | >{\centering\arraybackslash}X 
  | >{\centering\arraybackslash}X | }
\hline
 \textbf{Components}& \textbf{Values} & \textbf{Quantity}\\
\hline
Arduino & UNO & 1 \\  
\hline
JumperWires& M-M & 5 \\ 
\hline
Breadboard &  & 1 \\
\hline
\end{tabularx}
   \section{Implementation}
   	\subsection{METHOD-1}
   	\paragraph{}
The truth table  for Fig. 1 is available in Table-1
Using Boolean logic, output F in Table 1 can be expressed in terms of the inputs X, Y, Z as F=(X'+Y)+(Y.Z').................(2.1)
Built in led at 13th pin of Arduino will glow for the logic '1' of F based on the initialization of X,Y,Z.
The code below realizes the Boolean logic for F in Table-1
\\
\begin{lstlisting}
https://github.com/velicharlagokulkumar/FWC_module1/blob/main/Assignment-1/codes/method_1.ino
\end{lstlisting}
	\begin{tabularx}{0.46\textwidth} { 
  | >{\centering\arraybackslash}X 
  | >{\centering\arraybackslash}X 
  | >{\centering\arraybackslash}X
  | >{\centering\arraybackslash}X | }
\hline
 X & Y & Z & F \\
\hline
0 & 0 & 0 & 1 \\  
\hline
0 & 0 & 1 & 1 \\ 
\hline
0 & 1 & 0 & 1 \\
\hline
0 & 1 & 1 & 1 \\
\hline
1 & 0 & 0 & 0 \\  
\hline
1 & 0 & 1 & 0 \\ 
\hline
1 & 1 & 0 & 1 \\
\hline
1 & 1 & 1& 1 \\
\hline
\end{tabularx}
\begin{center}
Table-1 
  \end{center}
\begin{center}
\subsection{METHOD-2}
     \begin{karnaugh-map}[4][2][1][$YZ$][$X$]
        \minterms{0,1,2,3,6,7}
        \maxterms{4,5}
        \implicant{0}{2}
        \implicant{3}{6}
    \end{karnaugh-map}
\end{center}
\begin{center}
Fig. 2
\end{center}
    \paragraph{Karnugh Map :}
The expression in (2.1) can be minimized using the K-map in Fig 2. In Fig.2 ,the implicants in boxes 0,1,2,3 result in X'
The implicants in boxes 2,3,6,7 result in Y
Thus, after minimization using Fig. 2, (2.1) can
be expressed as
F=X'+ Y........(2.2).
Verify the truth table for F in TABLE 1.
The code below realizes the Boolean logic for F in 2.2
\begin{lstlisting}
https://github.com/velicharlagokulkumar/FWC_module1/blob/main/Assignment-1/codes/method_2.ino
\end{lstlisting}
\subsection{METHOD-3}
The code below realizes the Boolean logic for F in (2.2)  using 5V,GND of Arduino
\\
D3,D4,D5 Pins of Arduino are configured as input pins instead of initializing X,Y,Z inside software,inputs are given manually as X,Y,Z.Built in led will glow based on F satisfying the Table-1
\begin{lstlisting}
https://github.com/velicharlagokulkumar/FWC_module1/blob/main/Assignment-1/codes/method_3.ino
\end{lstlisting}
\bibliographystyle{ieeetr}
\end{document}